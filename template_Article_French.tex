% !TXS template
\documentclass[titlepage]{article}
\usepackage[utf8]{inputenc}
\usepackage{color}   %May be necessary if you want to color links
\usepackage{hyperref}
\usepackage[french]{babel}
\usepackage{cite}
\usepackage{graphicx}

\hypersetup{
	colorlinks=true, %set true if you want colored links
	linktoc=all,     %set to all if you want both sections and subsections linked
	linkcolor=blue,  %choose some color if you want links to stand out
}
\title{Arbre comportementaux pour le contr\^ole de mission}
\author{
	Derrien Martial \\
	\and
	Madec Antoine \\
	\and
	Lucas Florian
}
\date{Juin 2019}
\renewcommand{\contentsname}{Sommaire} 
\begin{document}
	\maketitle
	\tableofcontents
	\hypersetup{linktocpage}
	
	\clearpage
	\part{Introduction}
	Un arbre comportemental (Behavior Tree, BT en anglais) est une
	technique permettant de structurer le passage d'une tâche à une autre dans 
	le cadre de l'automatisation de ce comportement.
	\\
	Cette technique, d'abord utilisé pour programmer le comportement de personnages non joueurs (NPC) dans les jeux vidéos \cite{wikipedia_BT}, est maintenant l'une des techniques les plus en vogue pour la programmation de robot autonome ou semi-autonome\cite{ros.org}.
	\\
	\begin{figure}[h!]
		\includegraphics[width=\linewidth]{img/videogame_tree.jpg}
		\caption{Conception d'un jeu vidéo utilisant les BT}
		\label{fig:BT2}
	\end{figure}
	\\
	L'objectif de ce rapport est de faire une synthèse de l'utilisation émergente des BT en robotique, et ce dans de multiples domaines.
	\\
	Tout d'abord, nous allons définir ce qu'est un BT et quelles sont ses utilisations historiques. Ensuite, nous verrons qu'elles sont les nouvelles applications de cette technique dans les domaines variés de la robotique. Enfin, nous décrirons une implémentation simple d'un BT dans un projet de robotique théorique.
	\clearpage
	\part{Qu'est-ce-qu'un behavior tree ?}
	\section{définition}
	todo
	\\
	\begin{figure}[h!]
		\includegraphics[width=\linewidth]{img/behavior_trees_example.png}
		\caption{exemple de BT \cite{rasmussen}}
		\label{fig:BT1}
	\end{figure}
	\\
	\section{historique}
	todo
	\\
	\begin{figure}[h!]
		\includegraphics[width=\linewidth]{img/halo2.jpg}
		\caption{Le jeu vidéo Halo2, un précurseur en matière d'intelligence artificielle dans le jeu vidéo grâce à son utilisation des BT \cite{wikipedia_halo}}
		\label{fig:BT1}
	\end{figure}
	\\
	
	\clearpage
	\part{Les grands domaines d'emploi des behavior trees}
	\section{civil}
	
	\section{industrie}
	
	\section{santé}
	prédire comportement
	
	\section{militaire}
	prédire comptmt
	
	\clearpage
	\part{Les outils utilisés dans la mise en place des behavior trees}
	\section{machine learning}
	
	\clearpage
	\part{Exemple de behavior tree}
	
	\clearpage
	\bibliographystyle{plain}
	\bibliography{sources.bib}
	
\end{document}
