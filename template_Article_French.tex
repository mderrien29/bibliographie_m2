% !TXS template
\documentclass[titlepage]{article}
\usepackage[utf8]{inputenc}
\usepackage{color}   %May be necessary if you want to color links
\usepackage{hyperref}
\usepackage[french]{babel}
\usepackage{cite}
\usepackage{graphicx}

\hypersetup{
	colorlinks=true, %set true if you want colored links
	linktoc=all,     %set to all if you want both subsections and subsubsections linked
	linkcolor=blue,  %choose some color if you want links to stand out
}
\title{Arbre comportementaux pour le contr\^ole de mission}
\author{
	Derrien Martial \\
	\and
	Madec Antoine \\
	\and
	Lucas Florian
}
\date{Juin 2019}
\renewcommand{\contentsname}{Sommaire} 
\begin{document}
	\maketitle
	\tableofcontents
	\hypersetup{linktocpage}
	
	\clearpage
	\section{Introduction}
	Un arbre comportemental (Behavior Tree, BT en anglais) est une
	technique permettant de structurer le passage d'une tâche à une autre dans 
	le cadre de l'automatisation de ce comportement.
	\\
	Cette technique, d'abord utilisé pour programmer le comportement de personnages non joueurs (NPC) dans les jeux vidéos \cite{wikipedia_BT}, est maintenant l'une des techniques les plus en vogue pour la programmation de robot autonome ou semi-autonome\cite{ros.org}.
	\\
	\begin{figure}[h!]
		\includegraphics[width=\linewidth]{img/videogame_tree.jpg}
		\caption{Conception d'un jeu vidéo utilisant les BT}
		\label{fig:BT2}
	\end{figure}
	\\
	L'objectif de ce rapport est de faire une synthèse de l'utilisation émergente des BT en robotique, et ce dans de multiples domaines.
	\\
	Tout d'abord, nous allons définir ce qu'est un BT et quelles sont ses utilisations historiques. Ensuite, nous verrons qu'elles sont les nouvelles applications de cette technique dans les domaines variés de la robotique. Enfin, nous décrirons une implémentation simple d'un BT dans un projet de robotique théorique.
	\clearpage
	\section{Qu'est-ce-qu'un behavior tree ?}
		\subsection{Définition}
		Un arbre de comportement est un modèle mathématique d'exécution de plans utilisé en informatique, en robotique, en systèmes de contrôle et dans les jeux vidéo. Ils décrivent les basculements entre un ensemble fini de tâches de manière modulaire. Leur force provient de leur capacité à créer des tâches très complexes composées de tâches simples, sans se soucier de la façon dont les tâches simples sont mises en œuvre. \cite{wikipedia_BT}
		\subsection{fonctionnement}
		
		D'un point de vue conceptuel, un BT est basé sur deux objets clés : le nœud et l'arborescence.
		
		\begin{figure}[h!]
			\includegraphics[width=\linewidth]{img/behavior_trees_example.png}
			\caption{exemple de BT \cite{rasmussen}}
			\label{fig:BT1}
		\end{figure}
		
		\paragraph{Nœud}
		Le nœud est le concept le plus fondamental ici. il s'agit d'un bloc de construction qui peut être associé à d'autres pour construire des comportements. Un nœud se compose d'un bloc de code qui représente une tâche simple. Tous les nœuds ont la même interface : lors de leur traitement, ils effectuent une tâche et peuvent réussir ou échouer.
		
		Les nœuds peuvent être autonomes ou avoir des nœuds enfants, lesquels sont traités dans le cadre du traitement du nœud parent. Lors du traitement, la réussite d'un nœud parent dépend souvent (mais pas toujours) de la réussite de chaque nœud enfant.
		
		Les nœuds suivent plusieurs modèles courants, tels que les nœuds d'action, composites et décorateurs. \cite{documentation_aws}
		
		\paragraph{Arborescence}
		Les comportements sont élaborés en construisant des arborescences de nœuds, collections de tâches individuelles qui, lorsqu'elles sont positionnées comme une racine avec des branches qui se terminent par des feuilles, définissent la façon dont un agent intelligent se comportera en réponse à une entrée. \cite{documentation_aws}	
		\subsection{Behavior Tree face à FSM et HFSM}
		Une machine à états finis (FSM) est un modèle mathématique de calcul. C'est une machine abstraite qui peut se trouver exactement dans un nombre fini d'états à un moment donné. Le FSM peut passer d'un état à un autre en réponse à certaines entrées externes; le passage d'un état à un autre s'appelle une transition. Un FSM est défini par une liste de ses états, son état initial et les conditions de chaque transition.
		\\
		D'un point de vue théorique, chaque exécution décrite par un BT peut être décrite par un FSM et inversement. Toutefois, en raison du nombre de transitions, l'utilisation d'un FSM en tant qu'architecture de contrôle n'est pas pratique pour certaines applications robotiques. De plus, un FSM suppose que les propositions qui déclenchent les transitions sortantes à partir du même état s’excluent mutuellement. Une fois mises en œuvre, les propositions sont vérifiées régulièrement en un temps discret. Il existe donc une probabilité que deux propositions ou plus se tiennent simultanément après un cycle. Pour résoudre ce problème, il faut redéfinir la signification de certaines transitions, afin de les rendre mutuellement exclusives. Un FSM de ce format est peu pratique à concevoir pour les humains et les ordinateurs. Ajouter et supprimer des comportements humains est sujet à des erreurs. Après avoir ajouté un nouvel état, chaque transition existante doit être réévaluée et les nouvelles transitions vers le nouvel état doivent également être évaluées. Un grand nombre de transitions rend tout processus automatisé d'analyse ou de synthèse coûteux en FSM.
		\\
		Les marchines à états finis hierarchique (HFSM) sont les autorités de certification les plus similaires aux BT en termes d'objectif et d'expressivité. Pour comparer les BT aux HFSM, veuillez vous référer à \cite{colledanchise_2017}. Une différence importante est que, dans les HFSM, chaque couche de la hiérarchie doit être ajoutée explicitement, alors que dans les BT, chaque sous-arbre peut être vu comme un module à part entière.
		\subsection{historique}
			Les BT sont apparuent dans l'industrie des jeux vidéos car les FSM deviennent de plus en plus difficiles à déboguer. Damian Isla a expliqué cela en détail à propos de l’intelligence artificielle de Halo 2 lors de son exposé sur la GDC en 2005. \cite{gdc_2005}
			\\
			\begin{figure}[h!]
				\includegraphics[width=\linewidth]{img/halo2.jpg}
				\caption{Le jeu vidéo Halo2, un précurseur en matière d'intelligence artificielle dans le jeu vidéo grâce à son utilisation des BT \cite{wikipedia_halo}}
				\label{fig:BT1}
			\end{figure}
			\\
			Pour atténuer certains de ces problèmes, les FSM ont évolué en structures hiérarchiques facilitant la conception et le débogage. L'idée des hiérarchies étend la capacité de gérer de nombreux États, mais ne supprime pas le problème selon lequel, tôt ou tard, les FSM deviennent ingérables à mesure qu'ils grandissent en taille et en complexité.
			\\
			Unreal Engine a une belle implémentation d'arborescence de comportement qui sert de référence pour les arborescences de comportement dans l'IA du jeu.
			Les BT fournissent dans de nombreux cas un cadre pour la conception d'IA plus compréhensibles et plus faciles à lire que les FSM hiérarchiques. En outre, l’arbre bien organisé facilite le débogage visuel dans la pratique.
		

	
	\clearpage
	\section{Les behavior tree en robotique}
		\subsection{Civil}
		\paragraph{Exemple}
		\begin{figure}[h!]
			\includegraphics[width=\linewidth]{img/vehicul.jpg}
			\caption{Un poids lourd entièrement autonome iQmatic}
			\label{fig:civil}
		\end{figure}	
		iQmatic \cite{kth} est un projet dirigé par Scania (constructeur de camions haut de gamme) qui vise à développer un véhicule (camion, autobus, etc.) pour le transport de marchandises, l’exploitation minière et d’autres applications industrielles. 
		\\
		Le logiciel du véhicule doit être réutilisable, maintenable et facile à développer. Pour ces raisons, les développeurs d’iQmatic ont choisi BT comme architecture de contrôle pour le projet.
		Les BT sont appréciés dans iQmatic pour leur lisibilité humaine, qui supportent la conception et le développement des premiers prototypes. La figure 4 montre un des camions utilisés dans le banc d’essai iQmatic
		\subsection{Industrie}
		\paragraph{Exemple}
		\begin{figure}[h!]
			\includegraphics[width=\linewidth]{img/robotAma.JPG}
			\caption{The KTH project}
			\label{fig:civil}
		\end{figure}
		L’Amazon Picking Challenge (APC) \cite{apc} est une compétition internationale de robots.
		On demande aux robots de récupérer de façon autonome une vaste gamme de produits sur une tablette et de les mettre dans un bac. Le défi a été conçu dans le but de renforcer les liens entre la recherche robotique industrielle et universitaire et également de promouvoir des solutions communes à certains problèmes ouverts dans l’automatisation non structurée.
		\\
		L'équipe KTH a utilisé BT dans les deux éditions (2015 et
		2016). Les BT ont été appréciés pour leur modularité et leur réutilisation de code, ce qui a permis l’intégration de fonctionnalités différentes développées par les programmeurs avec différent styles de codage.
		\subsection{Santé}
		
		L’utilisation de la thérapie assistée par robot (RAT) dans les interventions de soins de santé a de plus en plus attiré l’attention de la recherche. Cependant, beaucoup d’études RAT sont menées sous les techniques du Magicien d’Oz (Woz) dans lesquelles les robots sont téléopérés ou préprogrammés. La tendance du RAT se déplace vers (partiellement) le contrôle autonome dans lequel l’architecture de contrôle du comportement des robots joue un rôle important dans la création d’une interaction humaine-robot efficace en engageant et en motivant les utilisateurs humains dans les processus thérapeutiques. \cite{rat} 
		Cela pourrait par exemple permettre un gain de temps dans les hôpitaux. 
	
		\subsection{militaire}
	prédire comptmt
	
	\clearpage
	\section{Les outils implémentant les behavior trees}
		En examinant les logiciels disponibles pour la conception et l'exécution de FSM et de BT, nous constatons que les outils du côté FSM, tels que IBM Rhapsody 2 et Stateflow 3, sont beaucoup plus avancés. Néanmoins, de nombreuses plates-formes de développement de jeux informatiques, telles que Unity3d 4 et Unreal Engine 5, disposent désormais d'outils permettant de travailler avec des BT. Pour ceux qui souhaitent mettre en œuvre leur propre cadre, nous notons que la mise en œuvre standard des FSM est assez simple, alors que les HFSM et les BT nécessitent davantage de considération. Cependant, les implémentations open source sont disponibles pour les deux.
		\subsection{outils utilisés}
		\subsection{outils incomplet}
		\subsection{machine learning}
		Dans la définition originale de l’arbre de comportement, les sélecteurs choisiront l’un de leurs nœuds-enfants à valider. L’algorithme parcourant l'arbre de gauche à droite pour trouver le premier des noeuds pouvant être activé donne aux noeuds de gauche les poids les plus élevés. \cite{Fu2016/08}
		\\
		Cependant, une considération plus naturelle consiste à attacher le poids aux nœuds enfants. Les sélecteurs cochent leurs nœuds enfants conformément aux poids de haut en bas. Cependant, cela implique d'ajuster manuellement les valeurs de pondération pour le système. Ce qui implique beaucoup de main-d'œuvre et de ressources : c'est une tâche très compliquée et très lourde, pas vraiment immaginable sur un très gros projet. \cite{Fu2016/08}
		\\
		Une solution serait donc d'utiliser l'apprentissage machine (machine learning) afin de faire ajuster ces valeurs automatiquement.
	\clearpage
	\section{Exemple d'implémentation}
	Pour notre exemple d'intégration, nous allons créer l'arbre comportemental d'un robot civil. Sa mission sera de remplacer les étudiants ayant du mal a se lever pour aller en cours. On ne s'intéressera pas à la conception (ni à la faisabilité) du dis robot, mais simplement la modélisation de son comportement. 
	\\
	\begin{figure}[h!]
		\centering
		\includegraphics[width=50px]{img/BT_1.png}
		\caption{Nous utiliserons les symboles suivants pour modéliser notre BT}
		\label{fig:exemple_1}
	\end{figure}
	\\
	Tout d'abord, nous modélisons les blocs principaux du comportement, les trois phases de la mission du robot.
	\begin{enumerate}
		\item Phase de démarrage
		\item Journée de cours
		\item Arret
	\end{enumerate}
	En partant de ces 3 séquences, nous avons modélisé un arbre :
	\begin{figure}[h!]
		\centering
		\includegraphics[width=\linewidth]{img/BT_2.png}
		\caption{Arbre après modélisation sommaire}
		\label{fig:exemple_2}
	\end{figure}
	L'une des forces des BT est la possibilité de modéliser les parties d'un arbre séparéments. Nous avons donc décider de s'attribuer chacun une partie différente.
	\\
	\paragraph{Démarrage}
	C'est l'arbre de démarrage, qui est executé au départ du robot. Pour éviter que le robot parte en cours en même temps que l'élève, il a été décidé d'ajouter une validation manuelle. Pour aussi éviter que le robot ne parte avec un faible niveau de batterie, et se retrouve bloqué, on ajoute aussi une condition de démarrage.
	\paragraph{Journée}
	C'est l'arbre principal du robot. Il décrit le comportement pendant la journée. C'est son arbre de mission. Il comporte une référence à l'action "aller en cours", qui sera décrit plus tard.
	\paragraph{Arrêt}
	Cet arbre est très simple pour nous. Il ordonne au robot de retourner à sa base, son poste de stockage ou de rechargement.
	\\
	\begin{figure}[h!]
		\centering
		\includegraphics[width=\linewidth]{img/BT_4.png}
		\caption{Arbre après modélisation des arbres principaux}
		\label{fig:exemple_3}
	\end{figure}
	\paragraph{Aller en cours}
	Cet arbre est l'arbre le plus important du comportement, car il décrit la partie la plus sensible de la mission du robot.
	\begin{figure}[h!]
		\centering
		\includegraphics[width=\linewidth]{img/BT_5.png}
		\caption{Arbre "Aller en cours"}
		\label{fig:exemple_4}
	\end{figure}
	\\
	Pour modéliser le comportement, nous avons fait appel à des "décorateurs", des noeuds qui modifient simplement la réponses du reste de l'arbre mais éxécutent quand même leur propre arbre. 
	\\
	Par exemple, le décorateur "BUSY" force l'abre inférieur à toujours répondre BUSY, ce qui permet au robot de ne pas éxecuter le reste de l'arbre. Il est utilisé s
	\\
	Nous avons aussi ajouté un arbre "changer de salle", qui est totalement optionel. Il  évalue quand même son arbre mais finit toujours par répondre "TRUE" ou succès car son éxecution n'est pas requise systématiquement.
	
	\clearpage
	\section{bibliographie}
	\bibliographystyle{plain}
	\bibliography{sources.bib}
	
\end{document}
